\fontfamily{pcr}\selectfont
\chapter{Implementation}
%replace the code with the one which associates with your project.

\section{Creating a User Login}
\begin{verbatim}
public void login () {
        // Choose authentication providers
        List<AuthUI.IdpConfig> providers = Arrays.asList(
                new AuthUI.IdpConfig.EmailBuilder().build(),
                new AuthUI.IdpConfig.PhoneBuilder().build(),
                new AuthUI.IdpConfig.GoogleBuilder().build()
        );
        // Create and launch sign-in intent
        Intent signInIntent = AuthUI.getInstance()
                .createSignInIntentBuilder()
                .setAvailableProviders(providers)
                .setLogo(R.drawable.logo)
                .setTheme(R.style.MealMingle)
                .build();
        signInLauncher.launch(signInIntent);
    }
\end{verbatim}
\section{Requesting User Logout}
\begin{verbatim}
public void logout (View v) {
        MainActivity act = this;
        AuthUI.getInstance()
                .signOut(this)
                .addOnCompleteListener(task -> {
                    // ...
//                        login();
                    Toast.makeText(act, "Logged Out Successfully !", 
                    Toast.LENGTH_SHORT).show();
                    loginButton.setVisibility(View.VISIBLE);
                    logoutButton.setVisibility(View.GONE);
                    ImageButton profilePic = 
                    findViewById(R.id.profileButton);
                    Picasso.get().load(R.drawable.baseline_account_circle_
                    24).error(R.drawable.baseline_account_circle_24).place
                    holder(R.drawable.baseline_account_circle_24).into(pro
                    filePic);

                });
    }
\end{verbatim}
\section{Configuring Account Type}
\begin{verbatim}
public void setAccountType(View v) {
        Intent intent = new Intent(this, ProfileActivityActivity.class);
//        startActivity(intent);
        // get button name
        String type = ((Button) v).getText().toString();

// set type in database
        UserManager.db.collection("users").document(user.getUid()).update(
        "type", type)
                .addOnCompleteListener(task -> {
                    if (task.isSuccessful()) {
                        Toast.makeText(this, "Account Type Updated 
                        Successfully !", Toast.LENGTH_SHORT).show();
                        startActivity(intent);
                    } else {
                        Toast.makeText(this, "Error Updating Account Type 
                        !", Toast.LENGTH_SHORT).show();
                    }
                });
    }
\end{verbatim}
\section{Utilizing Geohash for Meal Listings}
\begin{verbatim}
 public void listenToDatabase () {
        FirebaseFirestore db = FirebaseFirestore.getInstance();
        List<GeoQueryBounds> bounds = 
        GeoFireUtils.getGeoHashQueryBounds(center, radius);
        final List<Task<QuerySnapshot>> tasks = new ArrayList<>();
        for (GeoQueryBounds b : bounds) {
            Query q = db.collection("meals")
                    .orderBy("hash")
                    .startAt(b.startHash)
                    .endAt(b.endHash);

            tasks.add(q.get());
        }
\end{verbatim}
